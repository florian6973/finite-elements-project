On part de la formulation forte. On cherche le déplacement $\vec{u}$, fonction de $\mathbb{R}^3$ dans $\mathbb{R}^3$.

Pour cela, on réalise d'abord un bilan des forces sur le système étudié qui est le pont dans le référentiel terrestre. Le système est à l'équilibre donc $\vec{a} = 0$.

On définit

\begin{equation}
    \vec{f} = \begin{pmatrix} 0\\ 0\\ f\\\end{pmatrix}
\end{equation}

et sa valeur est spécifiée dans l'énoncé.

On définit 

\begin{equation}
    \vec{g} = \begin{pmatrix} 0\\ 0\\ -g\\\end{pmatrix}
\end{equation}

où $g$ est l'accélération de la pesanteur standard.

On se place en 3D. On définit le domaine $\Omega$ comme l'ensemble du pont (qui est un polygone), et la frontière $\Gamma$ est partitionnée comme selon le schéma.

Le schéma n'est pas à l'échelle.

D'après le cours de mécanique,

\begin{equation}
    \begin{cases}
        \lambda = \frac{E}{2(1+\nu)}\\
        \mu = \frac{\nu E}{(1+\nu)(1-2\nu)}
    \end{cases}
\end{equation}

et on cherche donc $\vec{u} \in \mathcal{C}^2(\bar{\Omega})^3$ tel que

\begin{equation}\label{fort}
    \begin{cases}
      \div \sigma + \rho \vec{g} = 0 \textrm { sur $\Omega$ (première loi de Cauchy)} \\
      \sigma = \lambda \Tr (\varepsilon) I + 2 \mu \varepsilon \textrm { (loi de comportement en élasticité linéaire)} \\
      \varepsilon=\frac{\nabla u + \nabla u^T }{2} \textrm { (compatibilité)} \\
      \sigma \cdot \vec{n} = \vec{f}\textrm { sur $\Gamma_F$ (condition de Neumann)} \\
      \sigma \cdot \vec{n} = 0\textrm { sur $\Gamma \backslash (\Gamma_F \cup \Gamma_S) $ (condition de Neumann)} \\
      \vec{u} = 0 \textrm { sur $\Gamma_S$ (condition de Dirichlet)} \\
      
    \end{cases}
\end{equation}

On a bien les conditions de continuité et de différentiabilité souhaitées. %à prouver, problème un peu différent%

Etant donné que l'on a dans le cours uniquement des formules pour des inconnues scalaires, on va écrire les équations pour chaque coordonnée pour arriver à la formulation variationnelle.

On note $\sigma_i$ la transposée de la $i$-ème ligne du $\sigma$.

Considérons une solution du problème.
Soit $i \in \{1,2,3\}$.
Soit $v_i \in \mathcal{C}^1(\bar{\Omega})$ tel que $v_{i|\Gamma_S} = 0$.
% def sigma
% def v


La première équation de \eqref{fort} se réécrit 
$$\div \vec{\sigma_i} + \rho g_i = 0$$
On multiplie par $v_i$ et on intègre sur $\Omega$ :
$$\int_\Omega v_i \div \vec{\sigma_i} +  \int_\Omega v_i \rho g_i = 0$$

On réalise une intégration par partie selon la formule de Green, car $\vec{\sigma_i} \in \mathcal{C}^1(\bar{\Omega})^3$ et $v_i \in \mathcal{C}^1(\bar{\Omega})$ :

$$-\int_{\Omega} \vec{\sigma_i} \cdot \grad v_i + \int_{\Gamma} v_i (\vec{\sigma_i} \cdot \vec{n}) + \int_\Omega v_i \rho g_i = 0$$

On injecte les conditions de Neumann et de Dirichlet dans l'équation : %+conditions de dirichlet 

\begin{equation}
    \begin{cases}
        -\displaystyle\int_{\Omega} \vec{\sigma_i} \cdot \grad v_i + \int_{\Gamma_F} v_i f_i + \int_\Omega v_i \rho g_i = 0\\
        u_i = 0 \textrm{ sur } \Gamma_S
    \end{cases}
\end{equation}

Sommons maintenant les trois équations.



On définit :

$$
\epsilon(\vec{x})=
\begin{pmatrix}
    \partial_1 x_1\\
    \partial_2 x_2\\ 
    \partial_3 x_3\\
    \frac{1}{\sqrt{2}}(\partial_3 x_2 + \partial_2 x_3)\\
    \frac{1}{\sqrt{2}}(\partial_3 x_1 + \partial_1 x_3)\\
    \frac{1}{\sqrt{2}}(\partial_2 x_1 + \partial_1 x_3)\\
\end{pmatrix}
$$

On arrive donc au problème suivant, après quelques calculs pour mettre réécrire l'équation avec la fonction précédemment définie :

Trouver $\vec{u} \in \mathcal{C}^2(\bar{\Omega})^3$ tel que
\begin{equation}
    \begin{cases}
        \displaystyle\int_{\Omega} (\lambda \div \vec{u} \div \vec{v} + 2\mu \epsilon(\vec{u})^T \epsilon(\vec{v})) - \int_{\Gamma_F} v f - \int_\Omega v_i \rho g = 0 \textrm{ } \forall \vec{v} \in \mathcal{C}^1(\bar{\Omega})^3\\
        \vec{u} = 0 \textrm{ sur } \Gamma_S
    \end{cases}
\end{equation}
