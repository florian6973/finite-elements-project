%marqueur aux points de données
% courbe en dessous/en dessus / intervalle dichotomie / condition d'arrêt / eps essai-erreur

Dans cette partie, nous résolvons le problème variationnel avec l'épaisseur fixée à $e=0.1$ m, à l'aide d'un programme FreeFem++.

% link free fem

\begin{problem}{3}
    Modélisons le pont, maillons-le et effectuons une simulation.
\end{problem}


    En suivant les indications de l'énoncé, nous avons commencé par réaliser les bordures, correctement "orientées" et "subdivisées", de la coupe du pont (selon l'axe $y$),
    selon la méthode vue en cours sur les maillages.%faire varier n
    Nous avons ensuite maillé la surface à l'aide de la fonction \emph{buildmesh} de FreeFem++.
    On note $n$ le paramètre contrôlant le nombre de subdivisions pour une bordure du maillage.

    \begin{figure}      
        \begin{center}
        
            \includegraphics[width=12cm]{imgs/all_maillage_default.PNG}
            \caption{Maillage basique du pont}
            \label{fig:maillage_default}
        
        \end{center}
    \end{figure}

    On remarque cependant sur la figure \ref{fig:maillage_default} que la zone d'intérêt a un maillage très grossier, tandis que les pieds du pont sont très maillés. 
    On choisit donc de pondérer les subdivisions du maillage ($n/2$, $n$ et $2n$), ce qui donne la figure \ref{fig:maillage_pondere}.
    
    \begin{figure}     
        \begin{center}
        
            \includegraphics[width=12cm]{imgs/all_maillage_pondere.PNG}
            \caption{Maillage plutôt homogène du pont}
            \label{fig:maillage_pondere}
        
        \end{center}
    \end{figure}

    Ensuite, on donne de la profondeur afin de passer à un maillage 3D avec la fonction \emph{buildlayers} de FreeFem++, 
    avec un paramètre $m=n/2$ qui est le nombre de subdivisions de la bordure du maillage. On effectue après une rotation de la structure 
    avec la fonction \emph{movemesh3}, en pensant à spécifier \emph{orientation=-1} pour ne pas avoir un maillage non conforme.
    On obtient ainsi la figure \ref{fig:maillage_3d}.

    
    \begin{figure}        
        \begin{center}
        
            \includegraphics[width=6cm]{imgs/all_maillage_3d.PNG}
            \caption{Maillage 3D du pont}
            \label{fig:maillage_3d}
        
        \end{center}
    \end{figure}

    Après écriture de l'équation \ref{varia} dans le langage de FreeFem++, on obtient le déplacement en P en cherchant la déformation verticale minimum, ainsi que 
    l'aperçu de la structure déformée (Figure \ref{fig:simu_def}). 
    Sur les figures, $coef$ désigne le coefficient d'amplification du déplacement pour pouvoir l'observer, $dep$ le déplacement en P et $lateral$ 
    le déplacement maximal selon l'axe $y$ (tout est en mètres).
    
    \begin{figure}        
        \begin{center}
        
            \includegraphics[width=6cm]{imgs/all_simu.PNG}
            \includegraphics[width=12cm]{imgs/all_simu_label.PNG}
            \caption{Résultat d'une simulation de toute la structure}
            \label{fig:simu_def}
        
        \end{center}
    \end{figure}

    On peut réaliser la même démarche pour la moitié de la structure en utilisant l'équation \ref{moitie} (Figure \ref{fig:simu_def_moitie}).

    \begin{figure}        
        \begin{center}
        
            \includegraphics[width=6cm]{imgs/half_simu.PNG}
            \includegraphics[width=12cm ]{imgs/half_simu_label.PNG}
            \caption{Résultat d'une simulation de la moitié de la structure}
            \label{fig:simu_def_moitie}
        
        \end{center}
    \end{figure}


\begin{problem}{4}
    Retravaillons le maillage pour avoir de meilleurs résultats et vérifier les propriétés de convergence.
\end{problem}


    Nous pouvons remarquer que le déplacement n'est pas le même entre la moitié de la structure et sa totalité (33 cm par rapport 21 cm).
    Cela pourrait donc signaler que le maillage n'est pas suffisamment fin pour que la solution approchée est convergée vers la solution exacte.
    Nous allons donc utiliser trois techniques pour faire converger les fonctions trouvées :
    \begin{itemize}
        \item Un raffinement manuel du maillage.
        \item Un raffinement automatique du maillage dans le plan de coupe avec \emph{adaptmesh} et manuel en profondeur.
        \item Un raffinement entièrement automatique avec la fonction \emph{mshmet} et \emph{mmg3d}.
    \end{itemize}

    D'après la figure \ref{fig:cvg}, on en déduit que le déplacement réel doit être proche de 48 cm. On constate une petite différence d'asymptote avec l'adaptation manuelle,
    qui pourrait être dûe par exemple par à la lenteur de la convergence ou, de manière plus probable, à des paramètres dans les fonctions d'adaptations du maillage
    (semi-)automatiques pas suffisamment optimisés. Par exemple, diminuer le paramètre erreur $err$ de 0.001 à 0.0001 pourrait améliorer ces méthodes (mais cela aurait un
    coût temporel conséquent). On constate cependant que la vitesse de convergence avec les fonctions d'adaptation est beaucoup plus rapide avec les paramètres choisis.

    Pour la suite des simulations, on choisit donc un maillage manuel avec $n=30$ et $m=15$, qui allie bonnes performances (1 seconde de calcul) avec une précision qui semble satisfaisante.
    
    \begin{figure}        
        \begin{center}
        
            \includegraphics[width=16.5cm]{imgs/cvg.png}
            \caption{Résultats des simulations (structure entière)}
            \label{fig:cvg}
        
        \end{center}
    \end{figure}


    On effectue la même étude de convergence pour la demi-structure (figure \ref{fig:h_cvg})

    \begin{figure}        
        \begin{center}
        
            \includegraphics[width=16.5cm]{imgs/cvgH.png}
            \caption{Résultats des simulations (demi-structure)}
            \label{fig:h_cvg}
        
        \end{center}
    \end{figure}

    Le détail des techniques d'adaptation de maillage peut être trouvées dans le code, disponible en annexe \ref{source_code}.

    La figure \ref{fig:maillage_auto} présente une adaptation de maillage automatique.
    
    \begin{figure}        
        \begin{center}
        
            \includegraphics[width=7cm]{imgs/half_maillage_auto.PNG}
            \caption{Exemple de maillage adapté automatiquement (demi-structure)}
            \label{fig:maillage_auto}
        
        \end{center}
    \end{figure}


    Les simulations ont mises quelques minutes à être calculées. 
    Il aurait pu être intéressant d'utiliser \emph{FreeFem++-mpi} et \emph{MUMPS-mpi} afin d'accélérer les calculs sur nos PCs.
    Il serait également possible de comparer les temps de chaque méthode pour savoir laquelle employée afin d'avoir les meilleures performances avec la meilleure précision.
    Dans tous les cas, il faut spécifier à toutes ces fonctions les "bons" arguments pour obtenir des résultats satisfaisants.

%critique : pas d'adapatation de m

%adapt mesh très mauvais maillage

% parallelization freefem

%optimisation temps

% étude convergence

% problème 2d équivalent
% adaptation de maillage 3d mmg3d et mshmet
% mumps, parallèle autre exécutable freefem
% erreur de valeurd