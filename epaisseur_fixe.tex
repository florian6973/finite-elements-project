%marqueur aux points de données
% courbe en dessous/en dessus / intervalle dichotomie / condition d'arrêt / eps essai-erreur

Dans cette partie, nous résolvons le problème variationnel avec l'épaisseur fixée à $e=0.1$ m, à l'aide d'un programme FreeFem++.

% link free fem

\begin{problem}{3}
    Modélisons le pont et maillons-le.
\end{problem}


\begin{solution}   
    En suivant les indications de l'énoncé, nous avons commencé par réaliser les bordures, correctement "orientées" et "subdivisées", de la coupe du pont (selon l'axe $y$),
    selon la méthode vue en cours sur les maillages.%faire varier n
    Nous avons ensuite maillé la surface à l'aide de la fonction \emph{buildmesh} de FreeFem++.

\end{solution}
%optimisation temps

% étude convergence