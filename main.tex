
\documentclass{article}
\usepackage[margin=1in]{geometry} 
\usepackage[french]{babel}
\usepackage[T1]{fontenc}
\usepackage[utf8]{inputenc}
\usepackage{amsmath,amsthm,amssymb,amsfonts, fancyhdr, color, comment, graphicx, environ}
\usepackage{xcolor}
\usepackage{mdframed}
\usepackage[shortlabels]{enumitem}
\usepackage{indentfirst}
\usepackage{hyperref}
\usepackage{lastpage}
\usepackage{listingsutf8}
\usepackage{ff++listings}

\renewcommand{\footrulewidth}{0.8pt}
\hypersetup{
    colorlinks=true,
    linkcolor=blue,
    filecolor=magenta,      
    urlcolor=blue,
}

\pagestyle{fancy}

\newenvironment{problem}[2][Etape]
    { \begin{mdframed}[backgroundcolor=gray!20] \textbf{#1 #2} \\}
    {  \end{mdframed}}


\newenvironment{solution}{\textbf{Réponse}}

\lhead{L. Gomes et F. Pollet}
\rhead{ES Elements finis} 
\chead{\textbf{Optimisation de l'épaisseur d'un pont}}
\lfoot{D. Ryckelynck, N. Spillane, P.-H. Tournier}
\cfoot{Mines Paris}
\rfoot{\thepage/\pageref*{LastPage}}

\begin{document}

    \title{\Large ES Elements Finis  \\[0.5cm]
        \bf\Large Optimisation de l'épaisseur d'une structure tridimensionnelle}
\author{\large Leticia Gomes\\Florent Pollet \ \\}
\date{\large\today}

\makeatletter
    \begin{titlepage}
        \begin{center}
	   { \includegraphics[width=12cm]{imgs/mp_logo.png}}
	   {\ \\ \ \\}
        \vbox{}\vspace{2cm}
            {\@title }\\[1cm] 
            { \includegraphics[width=7cm]{imgs/cover.PNG}}\\[1cm]
            {\@author}

            {\large \ \\ Encadrants: \bf David Ryckelynck, Nicole Spillane et Pierre-Henri Tournier\\ \ \\}
            {\@date\\}

        \end{center}

    \end{titlepage}

        
    \begin{remerciements}
        Nous tenons à remercier D. Ryckelynck, N. Spillane et P.-H. Tournier pour leurs cours, leur encadrement lors de cette semaine PSL et leurs conseils dans la réalisation
        de ce projet et la rédaction de ce rapport !
    \end{remerciements}


    \tableofcontents


    \clearpage
\makeatother%not necessary but looks fancy

    % rappel du problème ? reprendre photo du sujet
    % référence à la fin
    % appendix pour code ?

    \section{Présentation du problème}
    Dans le cadre de ce projet, nous avons cherché à minimiser l'épaisseur d'un pont qui donne un déplacement vertical maximal de 10 cm au centre de la structure. Le déplacement est dû à l'application d'une force constante à la partie horizontale du pont. La géométrie de cette problème est tridimensionnelle et son coupe est montré dans la Figure 1. % ajouter la coupe 2d et referencier
    Les extrémités inferiures sont fixes et l'épaisseur de la pièce est $p = 0,5 m$.
    Les donnés fournis pour ce système sont:
    \begin {itemize}
    \item Structure homogène;
    \item **Module d'Young :** $E = 2,1x10^11 N m^(-2)$
    \item **Coefficient de Poisson :** $\nu = 0,3$
    \item **Masse Volumique :** $\rho = 7800 kg m^(-3)$
    \item **Force Imposée sur le bord supérieur :** $f= - 5x10^8 N m^(-2)
    \section{Formulation variationnelle}

    \begin{problem}{1}
    Mise sous équation
    \end{problem}
    
    \begin{solution}    
    Blabla
    \end{solution}

    \section{Résolution à épaisseur fixée}

    \section{Optimisation de l'épaisseur}

    \subsection{Approximation graphique}
    \subsection{Raffinement de la solution}
    
    \clearpage
    \appendix
    \section {Code source}

    Disponible sur ...

    \lstinputlisting[language=FreeFem]{simulation-v2.edp}


    %code + lien github
    
\end{document}
