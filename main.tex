
\documentclass{article}
\usepackage[margin=1in]{geometry} 
\usepackage[french]{babel}
\usepackage[T1]{fontenc}
\usepackage[utf8]{inputenc}
\usepackage{amsmath,amsthm,amssymb,amsfonts, fancyhdr, color, comment, graphicx, environ}
\usepackage{xcolor}
\usepackage{mdframed}
\usepackage[shortlabels]{enumitem}
\usepackage{indentfirst}
\usepackage{hyperref}
\usepackage{lastpage}
\usepackage{listingsutf8}
\usepackage{ff++listings}

\usepackage{amsmath}
\DeclareMathOperator{\Tr}{Tr}
\usepackage{physics}
\usepackage{amsfonts}

\renewcommand{\footrulewidth}{0.8pt}
\hypersetup{
    colorlinks=true,
    linkcolor=blue,
    filecolor=magenta,      
    urlcolor=blue,
}

\pagestyle{fancy}

\newenvironment{problem}[2][Etape]
    { \begin{mdframed}[backgroundcolor=gray!20] \textbf{#1 #2} \\}
    {  \end{mdframed}}


\newenvironment{solution}{\textbf{Réponse}}

\lhead{L. Gomes et F. Pollet}
\rhead{ES Elements finis} 
\chead{\textbf{Optimisation de l'épaisseur d'un pont}}
\lfoot{D. Ryckelynck, N. Spillane, P.-H. Tournier}
\cfoot{Mines Paris}
\rfoot{\thepage/\pageref*{LastPage}}

\begin{document}

    \title{\Large ES Elements Finis  \\[0.5cm]
        \bf\Large Optimisation de l'épaisseur d'une structure tridimensionnelle}
\author{\large Leticia Gomes\\Florent Pollet \ \\}
\date{\large\today}

\makeatletter
    \begin{titlepage}
        \begin{center}
	   { \includegraphics[width=12cm]{imgs/mp_logo.png}}
	   {\ \\ \ \\}
        \vbox{}\vspace{2cm}
            {\@title }\\[1cm] 
            { \includegraphics[width=7cm]{imgs/cover.PNG}}\\[1cm]
            {\@author}

            {\large \ \\ Encadrants: \bf David Ryckelynck, Nicole Spillane et Pierre-Henri Tournier\\ \ \\}
            {\@date\\}

        \end{center}

    \end{titlepage}

        
    \begin{remerciements}
        Nous tenons à remercier D. Ryckelynck, N. Spillane et P.-H. Tournier pour leurs cours, leur encadrement lors de cette semaine PSL et leurs conseils dans la réalisation
        de ce projet et la rédaction de ce rapport !
    \end{remerciements}


    \tableofcontents


    \clearpage
\makeatother%not necessary but looks fancy

    % rappel du problème ? reprendre photo du sujet
    % référence à la fin
    % appendix pour code ?

    \section{Présentation du problème}
    Intro intro sur des éléments finis, parler de l'utilité d'eux pour les calculs    
    Une application envisageable pour les calculs des éléments finis est l'optimisation d'épaisseur d'une structure tridimensionelle, ce qui est l'objectif de ce projet. La structure chosit pour cela est un pont dont nous avons cherché à minimiser l'épaisseur qui donne un déplacement vertical maximal de son centre de 10 cm. Celui-ce est dû à l'application d'une force constante à la surface horizontale supèrieure du pont. La géométrie de cette problème est tridimensionnelle et sa coupe verticale est montré dans la Figure 1. %referencier la figure
    \begin{center}
	{ \includegraphics[width=12cm]{coupe_2D-schema.JPG}}
	\end{center}
    % mettre la légende: Figure 1 - Coupe verticale de la géométrie du pont. Une force $f= - 5x10^8 N m^(-2)$ est appliquée à sa surface supérieure. Le déplacement maximale $d$ est montré en bleue. Le système est qui est soumis à la gravité $g$. Parler partie conditions limites ici!!!!!!!!!!! 
    La structure est considérée comme homogène avec les extrémités inferiures fixes et une épaisseur $p = 0,5 m$. Les donnés fournis pour ce système sont:
    
    \begin{itemize}
    \item **Module d'Young :** $E = 2,1x10^11 N m^(-2)$
    \item **Coefficient de Poisson :** $\nu = 0,3$
    \item **Masse Volumique :** $\rho = 7800 kg m^(-3)$
    \item **Force imposée sur le bord supérieur :** $f= - 5x10^8 N m^(-2)$
    \end{itemize}
    
    Nous avons divisé l'execution de ce projet dans trois parties :
    1. Établissement de la formulation variationelle pour ce problème ;
    2. Résolution à épaisseur fixée : nous avons écrit un programme avec la langage FreeFem++ pour résoudre le système à épaisseur fixe $e = 0,1 m$, avec une maille itialement bidimentionelle qui a été changé à tridimensionelle avec la fonction $buildlayers$. Nous avons étudié la différence entre des solutions calculés pour la pièce et pour son moitié et aussi la précision du résultat en variant le maillage.
    3. Optimisation de l'épaisseur : 
    	a. Approximation graphique : nous avons analysé la rélation entre l'épaisseur et le déplacement afin de trouver un intervalle où le déplacement est inférieur ou égal à 10 cm.
	b. Raffinement de la solution : nous avons utilisé la méthode de dichotomie pour résoudre le système.
    \section{Formulation variationnelle}

    \begin{problem}{1}
    Mise sous équation du problème pour pouvoir le résoudre avec les éléments finis.
    \end{problem}
    
    \begin{solution}   
        On part de la formulation forte. On cherche le déplacement $\vec{u}$, fonction de $\mathbb{R}^3$ dans $\mathbb{R}^3$.

Pour cela, on réalise d'abord un bilan des forces sur le système étudié qui est le pont dans le référentiel terrestre. Le système est à l'équilibre donc $\vec{a} = 0$.

On définit

\begin{equation}
    \vec{f} = \begin{pmatrix} 0\\ 0\\ f\\\end{pmatrix}
\end{equation}

et sa valeur est spécifiée dans l'énoncé.

On définit 

\begin{equation}
    \vec{g} = \begin{pmatrix} 0\\ 0\\ -g\\\end{pmatrix}
\end{equation}

où $g$ est l'accélération de la pesanteur standard.

On se place en 3D. On définit le domaine $\Omega$ comme l'ensemble du pont (qui est un polygone), et la frontière $\Gamma$ est partitionnée comme selon le schéma.

D'après le cours de mécanique des matériaux,

\begin{equation}
    \begin{cases}
        \lambda = \frac{E}{2(1+\nu)}\\
        \mu = \frac{\nu E}{(1+\nu)(1-2\nu)}
    \end{cases}
\end{equation}

et on cherche donc $\vec{u} \in \mathcal{C}^2(\bar{\Omega})^3$ tel que

\begin{equation}\label{fort}
    \begin{cases}
      \div \sigma + \rho \vec{g} = 0 \textrm { sur $\Omega$ (première loi de Cauchy)} \\
      \sigma = \lambda \Tr (\varepsilon) I + 2 \mu \varepsilon \textrm { (loi de comportement en élasticité linéaire)} \\
      \varepsilon=\frac{\nabla u + \nabla u^T }{2} \textrm { (compatibilité)} \\
      \sigma \cdot \vec{n} = \vec{f}\textrm { sur $\Gamma_F$ (condition de Neumann)} \\
      \sigma \cdot \vec{n} = 0\textrm { sur $\Gamma \backslash (\Gamma_F \cup \Gamma_S) $ (condition de Neumann)} \\
      \vec{u} = 0 \textrm { sur $\Gamma_S$ (condition de Dirichlet)} \\
      
    \end{cases}
\end{equation}

On a bien les conditions de continuité et de différentiabilité souhaitées. %à prouver, problème un peu différent%

Etant donné que l'on a dans le cours uniquement des formules pour des inconnues scalaires, on va écrire les équations pour chaque coordonnée pour arriver à la formulation variationnelle.

On note $\sigma_i$ la transposée de la $i$-ème ligne du $\sigma$.

Considérons une solution du problème.
Soit $i \in \{1,2,3\}$.
Soit $v_i \in \mathcal{C}^1(\bar{\Omega})$ tel que $v_{i|\Gamma_S} = 0$.
% def sigma
% def v


La première équation de \eqref{fort} se réécrit 
$$\div \vec{\sigma_i} + \rho g_i = 0$$
On multiplie par $v_i$ et on intègre sur $\Omega$ :
$$\int_\Omega v_i \div \vec{\sigma_i} +  \int_\Omega v_i \rho g_i = 0$$

On réalise une intégration par partie selon la formule de Green, car $\vec{\sigma_i} \in \mathcal{C}^1(\bar{\Omega})^3$ et $v_i \in \mathcal{C}^1(\bar{\Omega})$ :


\begin{equation}\label{green}
    -\int_{\Omega} \vec{\sigma_i} \cdot \grad v_i + \int_{\Gamma} v_i (\vec{\sigma_i} \cdot \vec{n}) + \int_\Omega v_i \rho g_i = 0
\end{equation}

On injecte les conditions de Neumann et de Dirichlet dans l'équation : %+conditions de dirichlet 

\begin{equation}
    \begin{cases}
        -\displaystyle\int_{\Omega} \vec{\sigma_i} \cdot \grad v_i + \int_{\Gamma_F} v_i f_i + \int_\Omega v_i \rho g_i = 0\\
        u_i = 0 \textrm{ sur } \Gamma_S
    \end{cases}
\end{equation}

Sommons maintenant les trois équations pour chaque $i \in \{1,2,3\}$.

En utilisant la convention de sommation d'Einstein (les indices répétés sont sommés, de $1$ à $3$), on remarque que :
$$
\vec{\sigma_i} \cdot \grad v_i = \sigma_{ij} \frac{\partial v_i}{\partial x_j}
$$
puis que, en notant $\delta$ le symbole de Kronecker et en utilisant les propriétés de la trace :
$$
\sigma_{ij} = \lambda \Tr(\grad \vec{u}) \delta_{ji} + \mu (\frac{\partial u_i}{\partial x_j} + \frac{\partial u_j}{\partial x_i})
$$

On utilise le fait que $\Tr(\grad \vec{u}) = \div \vec{u}$ pour arriver à 

$$
\vec{\sigma_i} \cdot \grad v_i = \lambda \div \vec{u} \div \vec{v} + \mu (\frac{\partial u_i}{\partial x_j} + \frac{\partial u_j}{\partial x_i})\frac{\partial v_i}{\partial x_j}
$$
%(\delta_{ji} \frac{\partial v_i}{\partial x_j}) 

On définit :

$$
\epsilon(\vec{x})=
\begin{pmatrix}
    \partial_1 x_1\\
    \partial_2 x_2\\ 
    \partial_3 x_3\\
    \frac{1}{\sqrt{2}}(\partial_3 x_2 + \partial_2 x_3)\\
    \frac{1}{\sqrt{2}}(\partial_3 x_1 + \partial_1 x_3)\\
    \frac{1}{\sqrt{2}}(\partial_2 x_1 + \partial_1 x_3)\\
\end{pmatrix}
$$

On constate enfin, en développant, que :
$$
(\frac{\partial u_i}{\partial x_j} + \frac{\partial u_j}{\partial x_i})\frac{\partial v_i}{\partial x_j} = 2\epsilon(\vec{u})^T\epsilon(\vec{v})
$$

On arrive donc au problème suivant qui est sous forme variationnelle (formulation faible)  :

Trouver $\vec{u} \in \mathcal{C}^2(\bar{\Omega})^3$ tel que
\begin{equation}\label{varia}
    \begin{cases}
        \displaystyle\int_{\Omega} (\lambda \div \vec{u} \div \vec{v} + 2\mu \epsilon(\vec{u})^T \epsilon(\vec{v})) - \int_{\Gamma_F} \vec{v} \cdot \vec{f} - \int_\Omega \rho \vec{v} \cdot \vec{g} = 0 \textrm{ } \forall \vec{v} \in \mathcal{C}^1(\bar{\Omega})^3\\
        \vec{u} = 0 \textrm{ sur } \Gamma_S
    \end{cases}
\end{equation}


    \end{solution}

    \section{Résolution à épaisseur fixée}

    \section{Optimisation de l'épaisseur}

    \subsection{Approximation graphique}
    \subsection{Raffinement de la solution}
    
    \clearpage
    \appendix
    \section {Code source}

    Disponible sur ...

    \lstinputlisting[language=FreeFem]{simulation-v2.edp}


    %code + lien github
    
\end{document}
