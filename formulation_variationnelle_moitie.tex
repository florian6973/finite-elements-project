

% rapport scolaire ou non, pas de barème, faire qqch de bien
%résultat numériques
Intéressons-nous maintenant à la moitié de la structure. 

En effet, on remarque un plan de symétrie $x=0$ (passant par P), 
et parvenir à simuler uniquement une demi-structure permettrait de gagner en temps de calcul et en mémoire pour un même jeu de paramètres.

On s'intéresse à la partie droite du pont. On note $\Gamma_B$ la surface du bord correspondant à la tranche du pont, à l'endroit de la coupure.

On souhaite déterminer les nouvelles conditions aux limites pour pouvoir obtenir la formulation variationnelle de ce problème.

Les conditions sont inchangées sauf sur $\Gamma_B$.

Tout d'abord, par symétrie, 
$$\forall (x,y,z) \in \mathbb{R}^3,
\begin{cases}
 u_1(x,y,z) = - u_1(-x,y,z)\\
 u_2(x,y,z) = u_2(-x,y,z)\\
 u_3(x,y,z) = u_3(-x,y,z)
\end{cases}
$$

En évaluant en $x=0$, il vient
$$\forall (y,z) \in \mathbb{R}^2, u_1(0,y,z) = 0$$

On en déduit que :

$$\forall (y,z) \in \mathbb{R}^2, \partial_2 u_1 (0, y, z) = \lim_{dy\rightarrow 0} \frac{u_1(0,y+dy,z)-u_1(0,y,z)}{dy}=0$$

et que, puisque $u_2$ est une fonction paire par rapport à la variable $x$, $\partial_1 u_2$ est une fonction impaire par rapport à $x$, 
donc nulle en $0$. On déduit des propriétés similaires pour la variable $z$ au lieu de $y$.

On a donc la condition suivante : $u_1 = 0$ sur $\Gamma_B$ et $\sigma \cdot \vec{e_1}=
\begin{pmatrix}
    \sigma_{xx}\\
    \sigma_{yx}\\
    \sigma_{zx}
\end{pmatrix}$

On n'a pas besoin de connaitre $\sigma_{xx}$ car on a déjà $u_1 = 0$ sur $\Gamma_B$. 

D'après la loi de comportement et les propriétés précédemment démontrées, en notant $C$ une constante, on trouve que :
 $$
 \forall (y,z) \in \mathbb{R}^2,
 \begin{cases}
    \sigma_{yx}(0, y, z) = C (\partial_2 u_1 + \partial_1 u_2) = 0\\
    \sigma_{zx}(0, y, z) = C (\partial_3 u_1 + \partial_1 u_3) = 0
 \end{cases}
$$

En repartant de \ref{green}, on arrive au problème suivant, en notant $\Omega_m$ la moitié de la structure étudiée, et en restreignant $\Gamma_F$ et $\Gamma_S$ à $\Omega_m$ :

Trouver $\vec{u} \in \mathcal{C}^2(\bar{\Omega_m})^3$ tel que
\begin{equation}\label{moitie}
    \begin{cases}
        \displaystyle\int_{\Omega_m} (\lambda \div \vec{u} \div \vec{v} + 2\mu \epsilon(\vec{u})^T \epsilon(\vec{v})) - \int_{\Gamma_F} \vec{v} \cdot \vec{f} - \int_{\Omega_m} \rho \vec{v} \cdot \vec{g} = 0 \textrm{ } \forall \vec{v} \in \mathcal{C}^1(\bar{\Omega_m})^3, \textrm{ tel que } v_{|\Gamma_S} = 0\\
        \vec{u} = 0 \textrm{ sur } \Gamma_S\\
        u_1 = 0 \textrm{ sur } \Gamma_B
    \end{cases}
\end{equation}

Pour prouver l'existence et l'unicité des solutions, nous pourrions utiliser le théorème de Lax-Milgram, après vérification des hypothèses.

Une autre approche du problème non traitée dans ce document serait de se ramener au problème 2D équivalent, ce qui pourrait simplifier et optimiser grandement les simulations,
tout en fournissant des approximations satisfaisantes pour la majorité des usages.
% relire schéma
